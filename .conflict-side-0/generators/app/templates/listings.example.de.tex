<%= heading2 %>{Quellcode}

<%- bexample %>
\Cref{lst:XML} zeigt XML-Quelltext.
\Cref{line:comment} enthält einen Kommentar.

\begin{lstlisting}[
  language=XML,
  caption={Beispiel-XML-Listing},
  label={lst:XML}]
<listing name="example">
  <!-- comment --> (* \label{line:comment} *)
  <content>not interesting</content>
</listing>
\end{lstlisting}
<%- eexample %>

Der zusätzliche Paramter \verb+float+ führt dazu, dass das Listing auch floated.
\Cref{lst:flXML} zeigt das gleitendede Listing.

<%- bexample %>
\begin{lstlisting}[
  % Es ist möglcih, die Abstände bei Bedarf einzustellen
  % aboveskip=2.5\baselineskip,
  % belowskip=-.8\baselineskip,
  float,
  language=XML,
  caption={Beispiel-XML-Listing -- gleitend},
  label={lst:flXML}]
<listing name="example">
  Floating
</listing>
\end{lstlisting}
<%- eexample %>

Es ist möglich auch JSON zu setzen, wie in \cref{lst:json} gezeigt.

<%- bexample %>
\begin{lstlisting}[
  float,
  language=json,
  caption={Beispiel-JSON-listing},
  label={lst:json}]
{
  key: "value"
}
\end{lstlisting}
<%- eexample %>

Java ist auch möglich -- \cref{lst:java}.

<%- bexample %>
\begin{lstlisting}[
  caption={Example Java listing},
  label=lst:java,
  language=Java,
  float]
public class Hello {
    public static void main (String[] args) {
        System.out.println("Hello World!");
    }
}
\end{lstlisting}
<%- eexample %>
