<%= heading2 %>{Source Code}

<%- bexample %>
\Cref{lst:XML} shows source code written in XML.
\Cref{line:comment} contains a comment.

\begin{lstlisting}[
  language=XML,
  caption={Example XML Listing},
  label={lst:XML}]
<listing name="example">
  <!-- comment --> (* \label{line:comment} *)
  <content>not interesting</content>
</listing>
\end{lstlisting}
<%- eexample %>

One can also add \verb+float+ as parameter to have the listing floating.
\Cref{lst:flXML} shows the floating listing.

<%- bexample %>
\begin{lstlisting}[
  % one can adjust spacing here if required
  % aboveskip=2.5\baselineskip,
  % belowskip=-.8\baselineskip,
  float,
  language=XML,
  caption={Example XML listing -- placed as floating figure},
  label={lst:flXML}]
<listing name="example">
  Floating
</listing>
\end{lstlisting}
<%- eexample %>

One can also typeset JSON as shown in \cref{lst:json}.

<%- bexample %>
\begin{lstlisting}[
  float,
  language=json,
  caption={Example JSON listing -- placed as floating figure},
  label={lst:json}]
{
  key: "value"
}
\end{lstlisting}
<%- eexample %>

Java is also possible as shown in \cref{lst:java}.

<%- bexample %>
\begin{lstlisting}[
  caption={Example Java listing},
  label=lst:java,
  language=Java,
  float]
public class Hello {
    public static void main (String[] args) {
        System.out.println("Hello World!");
    }
}
\end{lstlisting}
<%- eexample %>
