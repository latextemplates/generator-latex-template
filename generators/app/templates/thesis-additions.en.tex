<% if (documentclass == "ustutt") { -%>
\setlength{\evensidemargin}{-24pt}
\setlength{\oddsidemargin}{-24pt}

% Wrong spacing of the TOC should not appear after 3 latex runs - see https://tex.stackexchange.com/a/702375/9075
% (outdated) This is necessary if you have more than 9 sections/subsections/subsubsections
% \makeatletter
% \renewcommand\l@section{\@dottedtocline{1}{1.5em}{3em}}
% \renewcommand\l@subsection{\@dottedtocline{2}{1.5em}{4.3em}}
% \renewcommand\l@subsubsection{\@dottedtocline{3}{1.5em}{5.6em}}
% \makeatother

%% Inhaltsverzeichnis (Schrift, Aussehen) sowie weitere Verzeichnisse ====
\setcounter{secnumdepth}{3}    % Abbildungsnummerierung mit groesserer Tiefe
\setcounter{tocdepth}{2}       % Inhaltsverzeichnis mit groesserer Tiefe

<% if ((latexcompiler == "pdflatex") || (latexcompiler == "both")) { indent = '' -%>
<% if (latexcompiler == "both") { indent = '  ' -%>
\ifpdftex
<% } -%>
<%= indent %>% Align page sizes
<%= indent %>% Seems to be required somehow
<%= indent %>\setlength{\pdfpageheight}{\paperheight}
<%= indent %>\setlength{\pdfpagewidth}{\paperwidth}
<% if (latexcompiler == "both") { -%>
\fi
<% } -%>

<% } -%>
<% } -%>
% DM: line-breaking-description env vom daniel w.

% credit goes to daniel w. :-)
%% --- Descriptions with line breaks in labels ---------------------------------
\usepackage{calc}

\newcommand*\Descriptionlabel[1]
{\raisebox{0pt}[1ex][0pt]
  {\makebox[\labelwidth][1]
    {\parbox[t]{\labelwidth}
      {\hspace{0pt}\textbf{#1:}}}}}

\newcommand*\Descriptionlabelx[1]
{\parbox[t]{\textwidth}
  {\textbf{#1}\\\mbox{}}
}

\newenvironment{Description}
{\begin{list}{}{\let\makelabel\Descriptionlabelx
    \setlength\labelwidth{1em}
    \setlength\leftmargin{\labelwidth+\labelsep}}}
{\end{list}}


% globally change line spacing of lists
% paralist has suspended development since 10 years.
% enumitem has been updated 2011-09-28
\usepackage[inline]{enumitem}
\setlist{partopsep=0pt,itemsep=1pt}

%------------------------------------------------------------------------
% fquote Fancy Quotation environment
% supports empty/optional author

% Use \sloppy to make right-margin easier?
% Set picture units to be relative to font size (em)?
% Use begingroup to rest units afterwards?

\usepackage{xifthen}% provides \isempty test
\definecolor{quotemark}{gray}{0.7}

%fquote environment with author as optional parameter
%usage: \begin{fquote}quote\end{fquote} or \begin{fquote}[Author]quote\end{fquote}
\newenvironment{fquote}[1][]{%
  \newcommand{\fqauthor}{\relax}
  \ifthenelse{\isempty{#1}}
  	{}% do nothing
  	{\renewcommand{\fqauthor}{\hfill\textsc{--- #1}}}
  \vspace{1em}
  \begin{list}{}{%
      \setlength{\leftmargin}{0.2\textwidth}
      \setlength{\rightmargin}{0.2\textwidth}
    }
    \item[]%
    \begin{picture}(0,0)(0,0)
      \put(-15,-5){\makebox(0,0){%
	  \scalebox{4.5}{\textcolor{quotemark}{\bfseries``}}}%
      }
    \end{picture}\em\ignorespaces%
}{%
  \newline%
  \makebox[0pt][l]{\hspace{0.6\textwidth}%
  \begin{picture}(0,0)(0,0)
    \put(15,10){\makebox(0,0){%
	\scalebox{4.5}{\textcolor{quotemark}{\rmfamily\bfseries''}}}%
    }
  \end{picture}}%
  \fqauthor
  \end{list}
}

%German fquote
%  1 parameter for the author's name, may be empty ("{}")
%  guaranteed German quotes (works with lualatex and babel package)
%  usage: \begin{gfquote}{Author}quote\end{gfquote}
\newenvironment{gfquote}[1]{%
  \newcommand{\fqauthor}{\relax}
  \ifthenelse{\isempty{#1}}
   {}% do nothing
   {\renewcommand{\fqauthor}{\hfill\textsc{\textemdash #1}}}
  \vspace{1em}
  \begin{list}{}{%
      \setlength{\leftmargin}{0.2\textwidth}
      \setlength{\rightmargin}{0.2\textwidth}
    }
    \item[]%
    \begin{picture}(0,0)(0,0)
      \put(-15,-5){\makebox(0,0){%
   \scalebox{4.5}{\textcolor{quotemark}{\bfseries \glqq}}}%
      }
    \end{picture}\em\ignorespaces%
}{%
  \newline%
  \makebox[0pt][l]{\hspace{0.6\textwidth}%
  \begin{picture}(0,0)(0,0)
    \put(15,10){\makebox(0,0){%
 \scalebox{4.5}{\textcolor{quotemark}{\rmfamily\bfseries \grqq}}}%
    }
  \end{picture}}%
  \fqauthor
  \end{list}
}

% fix incompatibilities between KOMA and other packages, mainly float.
% should be loaded at the very end - see http://tex.stackexchange.com/a/156256/9075
\usepackage{scrhack}
