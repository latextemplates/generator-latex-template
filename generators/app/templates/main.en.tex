<% if (documentclass == 'lncs') { -%>
% This template has been tested with LLNCS DOCUMENT CLASS -- version 2.20 (10-Mar-2018)

<% } -%>
% !TeX spellcheck = en-US
% !TeX encoding = utf8
% !TeX program = <%= latexcompiler %>
<% if (requiresShellEscape) { -%>
% !TeX TXS-program:compile = txs:///<%= latexcompiler %>/[--shell-escape]
<% } -%>
% !BIB program = <%= bibtextool %>
% -*- coding:utf-8 mod:LaTeX -*-
<% if (documentclass == 'scientific-thesis') { -%>

% EN: The following package allows \\ at the title page
%     For more information see https://github.com/latextemplates/scientific-thesis-cover/issues/4
\RequirePackage{kvoptions-patch}
<% } -%>
<% if (documentclass == 'lncs') { -%>

% "a4paper" enables:
%
%  - easy print out on DIN A4 paper size
%
% One can configure a4 vs. letter in the LaTeX installation. So it is configuration dependend, what the paper size will be.
% This option  present, because the current word template offered by Springer is DIN A4.
% We accept that DIN A4 cause WTFs at persons not used to A4 in USA.
%
% "runningheads" enables:
%
%  - page number on page 2 onwards
%  - title/authors on even/odd pages
%
% This is good for other readers to enable proper archiving among other papers and pointing to
% content. Even if the title page states the title, when printed and stored in a folder, when
% blindly opening the folder, one could hit not the title page, but an arbitrary page. Therefore,
% it is good to have title printed on the pages, too.
%
% To disable outputting page headers and footers, remove "runningheads"
\documentclass[runningheads,a4paper,english]{llncs}[2018/03/10]
<% } else {-%>
\documentclass[
  % fontsize=11pt is the standard
  numbers=noenddot,
  english,  % English as main language; this parameter is passed to other packages (e.g., selnolig in the case of lualatex)
  a4paper,  % Standard format - only KOMAScript uses paper=a4 - https://tex.stackexchange.com/a/61044/9075
  twoside,  % we are optimizing for both screen and two-side printing. So the page numbers will jump, but the content is configured to stay in the middle (by using the geometry package)
  bibliography=totoc,
  %               idxtotoc,   %Index ins Inhaltsverzeichnis
  %               liststotoc, %List of X ins Inhaltsverzeichnis, mit liststotocnumbered werden die Abbildungsverzeichnisse nummeriert
  headsepline,
  cleardoublepage=empty,
  parskip=half,
  %               draft    % um zu sehen, wo noch nachgebessert werden muss - wichtig, da Bindungskorrektur mit drin
  draft=false
]{scrbook}
<% } -%>

% backticks (`) are rendered as such in verbatim environments.
% See following links for details:
%   - https://tex.stackexchange.com/a/341057/9075
%   - https://tex.stackexchange.com/a/47451/9075
%   - https://tex.stackexchange.com/a/166791/9075
\usepackage{upquote}

<%- include('babel.preamble.en.tex', this); %>
<%- include('url.preamble.en.tex', this); %>
<%- include('font.preamble.en.tex', this); %>
<%- include('microtype.preamble.en.tex', this); %>
\usepackage{graphicx}

<% if (enquotes == 'csquotes')  { %><%- include('csquotes.preamble.en.tex', this); %><% } -%>
<% if (enquotes == 'textcmds')  { %><%- include('textcmds.preamble.en.tex', this); %><% } -%>

<%- include('diagbox.preamble.en.tex', this); -%>

<%# Required for package pdfcomment later -%>
\usepackage{xcolor}

<%- include(listings + '.preamble.en.tex', this); %>
<%- include('booktabs.preamble.en.tex', this); %>
<%- include('paralist.preamble.en.tex', this); %>
<% if (documentclass == 'lncs')  { -%>
<%- include('floatflt.preamble.en.tex', this); %>
% Bibliopgraphy enhancements
%  - enable \cite[prenote][]{ref}
%  - enable \cite{ref1,ref2}
% Alternative: \usepackage{cite}, which enables \cite{ref1, ref2} only (otherwise: Error message: "White space in argument")

% Doc: http://texdoc.net/natbib
\usepackage[%
  square,        % for square brackets
  comma,         % use commas as separators
  numbers,       % for numerical citations;
%  sort,          % orders multiple citations into the sequence in which they appear in the list of references;
  sort&compress, % as sort but in addition multiple numerical citations
                 % are compressed if possible (as 3-6, 15);
]{natbib}
% In the bibliography, references have to be formatted as 1., 2., ... not [1], [2], ...
\renewcommand{\bibnumfmt}[1]{#1.}

% Prepare more space-saving rendering of the bibliography
% Source: https://tex.stackexchange.com/a/280936/9075
\SetExpansion
[ context = sloppy,
  stretch = 30,
  shrink = 60,
  step = 5 ]
{ encoding = {OT1,T1,TS1} }
{ }

<% } -%>
<% if (todo == 'pdfcomment') { -%>
<%- include('pdfcomment.preamble.en.tex', this); %>
<% } else { -%>
<%- include('plainlatextodo.preamble.en.tex', this); %>
<% } -%>

% Put footnotes below floats
% Source: https://tex.stackexchange.com/a/32993/9075
\usepackage{stfloats}
\fnbelowfloat

<%- include('siunitx.preamble.en.tex'); %>
<%- include('hyperref.preamble.en.tex', this); %>
<% if (cleveref) { %><%- include('cleveref.preamble.en.tex', this); %><% } -%>

<% if (useExampleEnvironment) { -%>
\usepackage{currfile}
\usepackage{tcolorbox}
\tcbuselibrary{<%= listings %>}
<% } -%>
<% if (examples) { -%>

% For demonstration purposes only
% These packages can be removed when all examples have been deleted
\usepackage[math]{blindtext}
\usepackage{mwe}
<% } -%>

<%- include('powerset.preamble.en.tex', this); %>

\usepackage{xspace}
%\newcommand{\eg}{e.\,g.\xspace}
%\newcommand{\ie}{i.\,e.\xspace}
\newcommand{\eg}{e.\,g.,\ }
\newcommand{\ie}{i.\,e.,\ }

% Enable hyphenation at other places as the dash.
% Example: applicaiton\hydash specific
\makeatletter
\newcommand{\hydash}{\penalty\@M-\hskip\z@skip}
% Definition of "= taken from http://mirror.ctan.org/macros/latex/contrib/babel-contrib/german/ngermanb.dtx
\makeatother

% correct bad hyphenation here
\hyphenation{op-tical net-works semi-conduc-tor}
<% if (documentclass == 'lncs') { -%>

% Add copyright
% Do that for the final version or if you send it to colleagues
\iffalse
  %state: intended|submitted|llncs
  %you can add "crop" if the paper should be cropped to the format Springer is publishing
  \usepackage[intended]{llncsconf}

  \conference{name of the conference}

  %in case of "llncs" (final version!)
  %example: llncs{Anonymous et al. (eds). \emph{Proceedings of the International Conference on \LaTeX-Hacks}, LNCS~42. Some Publisher, 2016.}{0042}
  \llncs{book editors and title}{0042} %% 0042 is the start page
\fi
<% } -%>
<% if (latexcompiler == "pdflatex") { -%>

% Enable copy and paste of text from the PDF
% Only required for pdflatex. It "just works" in the case of lualatex.
% Alternative: cmap or mmap package
% mmap enables mathematical symbols, but does not work with the newtx font set
% See: https://tex.stackexchange.com/a/64457/9075
% Other solutions outlined at http://goemonx.blogspot.de/2012/01/pdflatex-ligaturen-und-copynpaste.html and http://tex.stackexchange.com/questions/4397/make-ligatures-in-linux-libertine-copyable-and-searchable
% Trouble shooting outlined at https://tex.stackexchange.com/a/100618/9075
%
% According to https://tex.stackexchange.com/q/451235/9075 this is the way to go
\input glyphtounicode
\pdfgentounicode=1
<% } -%>

\begin{document}
<% if (documentclass == 'lncs') { -%>

\title{Paper Title}
% If Title is too long, use \titlerunning
%\titlerunning{Short Title}

% Single insitute
\author{Firstname Lastname \and Firstname Lastname}

% If there are too many authors, use \authorrunning
%\authorrunning{First Author et al.}

\institute{Institute}

%% Multiple insitutes - ALTERNATIVE to the above
% \author{%
%     Firstname Lastname\inst{1} \and
%     Firstname Lastname\inst{2}
% }
%
%If there are too many authors, use \authorrunning
%  \authorrunning{First Author et al.}
%
%  \institute{
%      Insitute 1\\
%      \email{...}\and
%      Insitute 2\\
%      \email{...}
%}

\maketitle

\begin{abstract}
  \lipsum[1]
\end{abstract}

\begin{keywords}
  keyword1, keyword2
\end{keywords}

\section{Introduction}
\label{sec:intro}
\lipsum[1-3]\todo{Refine me}

The remainder of the paper starts with a presentation of related work (\cref{sec:relatedwork}).
<% if (examples) { %>It is followed by a presentation of hints on \LaTeX{} (\cref{sec:latexhints}).
<% } -%>
Finally, a conclusion is drawn and outlook on future work is made (\cref{sec:outlook}).

\section{Related Work}
\label{sec:relatedwork}

Winery~\cite{Winery} is a graphical \commentontext{modeling}{modeling with one ``l'', because of AE} tool.
The whole idea of TOSCA is explained by <% if (available.citet) { %>\citet<% } else { %>Binz et al.~\cite<% } %>{Binz2009}.
<% } -%>

<% if (examples) { -%>

<%= heading1 %>{LaTeX Hints}
<% if (documentclass == 'lncs') { -%>
\label{sec:latexhints}
<% } else { -%>
\label{chap:latexhints}
<% } -%>

<% if (useExampleEnvironment) { -%>
% Required for proper example rendering in the compiled PDF
\newcount\LTGbeginlineexample
\newcount\LTGendlineexample
\newenvironment{ltgexample}%
{\LTGbeginlineexample=\numexpr\inputlineno+1\relax}%
{%
\LTGendlineexample=\numexpr\inputlineno-1\relax%

\tcbinputlisting{%
  listing only,
  listing file=\currfilepath,
  colback=green!5!white,
  colframe=green!25,
  coltitle=black!90,
  coltext=black!90,
  left=8mm,
  title=Corresponding \LaTeX{} code of \texttt{\currfilepath},
  listing options={%
    frame=none,
    language={[LaTeX]TeX},
    escapeinside={},
    firstline=\the\LTGbeginlineexample,
    lastline=\the\LTGendlineexample,
    firstnumber=\the\LTGbeginlineexample,
    basewidth=.5em,
    aboveskip=0mm,
    belowskip=0mm,
    numbers=left,
    xleftmargin=0mm,
    numberstyle=\tiny,
    numbersep=8pt%
  }
}
}%
<% } -%>

This <% if (heading1 == "\\chapter") {%>chapter<% } else { %>section<% } %> contains hints on writing LaTeX.
It focuses on minimal examples, which can be directly adapted to the content

<%- include('paragraphs.example.en.tex', this); %>
<%- include('hyphenation.example.en.tex', this); %>
<%- include('siunitx.example.en.tex', this); %>
<% if (enquotes == 'csquotes')  { %><%- include('csquotes.example.en.tex', this); %><% } -%>
<% if (enquotes == 'textcmds')  { %><%- include('textcmds.example.en.tex', this); %><% } -%>
<% if (enquotes == 'plainlatex')  { %><%- include('plainlatex.enquotes.example.en.tex', this); %><% } -%>
<%- include('cleveref.example.en.tex', this); %>
<%- include('figure.example.en.tex', this); %>
<%- include('tables.example.en.tex', this); %>
<% if (listings == 'listings') { -%>
<%- include('listings.example.en.tex', this); %>
<% } else { -%>
<%- include('minted.example.en.tex', this); %>
<% } -%>
<%- include('paralist.example.en.tex', this); %>
<%- include('otherfeatures.example.en.tex', this); %>
<% } -%>

<% if (documentclass == 'lncs') { -%>
\section{Conclusion and Outlook}
\label{sec:outlook}
\lipsum[1-2]

\subsubsection*{Acknowledgments}
\ldots

In the bibliography, use \texttt{\textbackslash textsuperscript} for <%- bquote %>st<%- equote %>, <%- bquote %>nd<%- equote %>, \ldots:
E.g., <%- bquote %>The 2\textsuperscript{nd} conference on examples<%- equote %>.
When you use \href{https://www.jabref.org}{JabRef}, you can use the clean up command to achieve that.
See \url{https://help.jabref.org/en/CleanupEntries} for an overview of the cleanup functionality.

\renewcommand{\bibsection}{\section*{References}} % requried for natbib to have "References" printed and as section*, not chapter*
% Use natbib compatbile splncsnat style.
% It does provide all features of splncs03, but is developed in a clean way.
% Source: http://phaseportrait.blogspot.de/2011/02/natbib-compatible-bibtex-style-bst-file.html
\bibliographystyle{splncsnat}
\begingroup
  \microtypecontext{expansion=sloppy}
  \small % ensure correct font size for the bibliography
  \bibliography{<%= filenames.bib %>}
\endgroup

% Enfore empty line after bibliography
\ \\
%
All links were last followed on October 5, 2020.
<% } -%>
\end{document}
