% !TeX spellcheck = en-US
% !TeX encoding = utf8
% !TeX program = <%= latexcompiler %>
% !BIB program = <%= bibtextool %>
% -*- coding:utf-8 mod:LaTeX -*-

<% if (documentclass === 'scientific-thesis') { -%>
% EN: The following package allows \\ at the title page
%     For more information see https://github.com/latextemplates/scientific-thesis-cover/issues/4
\RequirePackage{kvoptions-patch}
<% } -%>

\documentclass[
  % fontsize=11pt is the standard
  numbers=noenddot,
  a4paper,  % Standard format - only KOMAScript uses paper=a4 - https://tex.stackexchange.com/a/61044/9075
  twoside,  % we are optimizing for both screen and two-side printing. So the page numbers will jump, but the content is configured to stay in the middle (by using the geometry package)
  bibliography=totoc,
  %               idxtotoc,   %Index ins Inhaltsverzeichnis
  %               liststotoc, %List of X ins Inhaltsverzeichnis, mit liststotocnumbered werden die Abbildungsverzeichnisse nummeriert
  headsepline,
  cleardoublepage=empty,
  parskip=half,
  %               draft    % um zu sehen, wo noch nachgebessert werden muss - wichtig, da Bindungskorrektur mit drin
  draft=false
]{scrbook}

<% if (latexcompiler == "pdflatex") { -%>
% Enable copy and paste of text from the PDF
% Only required for pdflatex. It "just works" in the case of lualatex.
% mmap enables mathematical symbols, but does not work with the newtx font set
% See: https://tex.stackexchange.com/a/64457/9075
% Other solutions outlined at http://goemonx.blogspot.de/2012/01/pdflatex-ligaturen-und-copynpaste.html and http://tex.stackexchange.com/questions/4397/make-ligatures-in-linux-libertine-copyable-and-searchable
% Trouble shooting outlined at https://tex.stackexchange.com/a/100618/9075
\usepackage{cmap}
% TODO: check font
<% } -%>

% backticks (`) are rendered as such in verbatim environments.
% See following links for details:
%   - https://tex.stackexchange.com/a/341057/9075
%   - https://tex.stackexchange.com/a/47451/9075
%   - https://tex.stackexchange.com/a/166791/9075
\usepackage{upquote}

<%- include('babel.preamble.en.tex', this); %>
<%- include('url.preamble.en.tex', this); %>
<%- include('font.preamble.en.tex', this); %>
<%- include('microtype.preamble.en.tex', this); %>
<%- include('listings.preamble.en.tex', this); %>
<%- include('hyperref.preamble.en.tex', this); %>
<% if (cleveref) { %><%- include('cleveref.preamble.en.tex', this); %><% } -%>
<% if (useExampleEnvironment) { -%>
\usepackage{currfile}
\usepackage{tcolorbox}
\tcbuselibrary{listings}
<% } -%>

\begin{document}

<% if (examples) { -%>

<%= heading1 %>{LaTeX Hints}
\label{chap:latexhints}

<% if (useExampleEnvironment) { -%>
% Required for proper example rendering in the compiled PDF
\newcount\LTGbeginlineexample
\newcount\LTGendlineexample
\newenvironment{ltgexample}%
{\LTGbeginlineexample=\numexpr\inputlineno+1\relax}%
{%
\LTGendlineexample=\numexpr\inputlineno-1\relax%

\tcbinputlisting{%
  listing only,
  listing file=\currfilepath,
  colback=green!5!white,
  colframe=green!25,
  coltitle=black!90,
  coltext=black!90,
  left=8mm,
  title=Corresponding \LaTeX{} code of \texttt{\currfilepath},
  listing options={%
    language={[LaTeX]TeX},
    firstline=\the\LTGbeginlineexample,
    lastline=\the\LTGendlineexample,
    firstnumber=\the\LTGbeginlineexample,
    basewidth=.5em,
    aboveskip=0mm,
    belowskip=0mm,
    numbers=left,
    xleftmargin=0mm,
    numberstyle=\tiny,
    numbersep=8pt%
  }
}
}%
<% } -%>

This <% if (heading1 == "\\chapter") {%>chapter<% } else { %>section<% } %> contains hints on writing LaTeX.
It focuses on minimal examples, which can be directly adapted to the content

<%- include('paragraphs.example.en.tex', this); %>
<%- include('figure.example.en.tex', this); %>
<%- include('listings.example.en.tex', this); %>
<% } -%>

\end{document}
