% EN: Character protrusion and font expansion. See http://www.ctan.org/tex-archive/macros/latex/contrib/microtype/
% DE: Optischer Randausgleich und Grauwertkorrektur

\usepackage[
  babel=true, % EN: Enable language-specific kerning. Take language-settings from the languge of the current document (see Section 6 of microtype.pdf)
  expansion=alltext,
  protrusion=alltext-nott, % EN: Ensure that at listings, there is no change at the margin of the listing
  final % EN: Always enable microtype, even if in draft mode. This helps finding bad boxes quickly.
        %     In the standard configuration, this template is always in the final mode, so this option only makes a difference if "pros" use the draft mode
]{microtype}

% EN: \texttt{test -- test} keeps the "--" as "--" (and does not convert it to an en dash)
\DisableLigatures{encoding = T1, family = tt* }

% DE: fuer microtype
% DE: tracking=true muss als Parameter des microtype-packages mitgegeben werden
% DE: Deaktiviert, da dies bei Algorithmen seltsam aussieht

%\DeclareMicrotypeSet*[tracking]{my}{ font = */*/*/sc/* }%
%\SetTracking{ encoding = *, shape = sc }{ 45 }
% DE: Hier wird festgelegt,
%     dass alle Passagen in Kapitälchen automatisch leicht
%     gesperrt werden.
%     Quelle: http://homepage.ruhr-uni-bochum.de/Georg.Verweyen/pakete.html
%    Deaktiviert, da sonst "BPEL", "BPMN" usw. wirklich komisch aussehen.
%     Macht wohl nur bei geisteswissenschaftlichen Arbeiten Sinn.
