<%= heading2 %>{Source Code}

\href{https://github.com/gpoore/minted}{minted} is a sophisticated package to enable properly highlighted listings.
It uses the \href{http://pygments.org/}{pygments} library, which in turn requires Python.

<%- bexample %>
\Cref{lst:XML} shows source code written in XML.
\refline{line:comment} contains a comment.

\begin{listing}[htbp]
  \begin{minted}[linenos=true,escapeinside=||]{xml}
<listing name="example">
  <!-- comment --> |\labelline{line:comment}|
  <content>not interesting</content>
</listing>
\end{minted}
  \caption{Example XML listing using minted}
  \label{lst:XML}
\end{listing}
<%- eexample %>

One can also typeset JSON as shown in \cref{lst:flJSON}.

<%- bexample %>
\begin{listing}[htbp]
  \begin{minted}[linenos=true,escapeinside=||]{json}
{
  key: "value"
}
\end{minted}
  \caption{Example JSON listing using minted}
  \label{lst:flJSON}
\end{listing}
<%- eexample %>

Java is also possible as shown in \cref{lst:flJava}.

<%- bexample %>
\begin{listing}[htbp]
  \begin{minted}[linenos=true,escapeinside=||]{java}
public class Hello {
    public static void main (String[] args) {
        System.out.println("Hello World!");
    }
}
\end{minted}
  \caption{Java code rendered using minted}
  \label{lst:flJava}
\end{listing}
<%- eexample %>
