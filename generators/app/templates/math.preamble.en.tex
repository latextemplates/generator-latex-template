%math stuff
\usepackage[
  centertags,    % (default) center tags vertically
  % tbtags,        % 'Top-or-bottom tags': For a split equation, place equation numbers level with the last (resp. first) line, if numbers are on the right (resp. left).
  sumlimits,    % (default) Place the subscripts and superscripts of summation symbols above and below
  % nosumlimits,   % Always place the subscripts and superscripts of summation-type symbols to the side, even in displayed equations.
  intlimits,     % Like sumlimits, but for integral symbols.
  % nointlimits,   % (default) Opposite of intlimits.
  namelimits,    % (default) Like sumlimits, but for certain 'operator names' such as det, inf, lim, max, min, that traditionally have subscripts placed underneath when they occur in a displayed equation.
  % nonamelimits,  % Opposite of namelimits.
  % leqno,         % Place equation numbers on the left.
  % reqno,         % Place equation numbers on the right.
  fleqn,         % Position equations at a fixed indent from the left margin rather than centered in the text column.
]{amsmath}
\SetMathAlphabet{\mathcal}{normal}{OMS}{amsa}{m}{n} %% AMS font for mathcal

%%% Doc: http://mirror.ctan.org/tex-archive/macros/latex/contrib/mh/doc/mathtools.pdf
% Erweitert amsmath und behebt einige Bugs
\usepackage[fixamsmath,disallowspaces]{mathtools}

%%% Doc: http://www.ctan.org/info?id=fixmath
% LaTeX's default style of typesetting mathematics does not comply
% with the International Standards ISO31-0:1992 to ISO31-13:1992
% which indicate that uppercase Greek letters always be typeset
% upright, as opposed to italic (even though they usually
% represent variables) and allow for typesetting of variables in a
% boldface italic style (even though the required fonts are
% available). This package ensures that uppercase Greek be typeset
% in italic style, that upright $\Delta$ and $\Omega$ symbols are
% available through the commands \upDelta and \upOmega; and
% provides a new math alphabet \mathbold for boldface
% italic letters, including Greek.
\usepackage{fixmath}

%for theorems, replacement for amsthm
\usepackage[amsmath,hyperref]{ntheorem}
\theorempreskipamount 2ex plus1ex minus0.5ex
\theorempostskipamount 2ex plus1ex minus0.5ex
\theoremstyle{break}
\newtheorem{definition}{Definition}[chapter]

%%% Doc: http://mirror.ctan.org/tex-archive/macros/latex/contrib/onlyamsmath/onlyamsmath.dvi
% Warnt bei Benutzung von Befehlen die mit amsmath inkompatibel sind.

% Braucht man evtl. nicht.
% \usepackage[
% 	all,
% 	warning
% ]{onlyamsmath}
