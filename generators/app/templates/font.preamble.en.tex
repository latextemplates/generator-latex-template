<% if (latexcompiler == "lualatex") { -%>
\usepackage[no-math]{fontspec}
\usepackage{unicode-math}
  <% if (font == "arial") { -%>
\setmainfont{Arial}
  <% } else if (font == "times") { -%>
% See https://tug.org/FontCatalogue/texgyretermes/ for more information
\setmainfont{texgyretermes}[
  Extension = .otf,
  UprightFont = *-regular,
  BoldFont = *-bold,
  ItalicFont = *-italic,
  BoldItalicFont = *-bolditalic,
  Ligatures=TeX
]
% See https://tug.org/FontCatalogue/texgyreheros/ for more information
\setsansfont[Scale=.9]{TeX Gyre Heros Regular}
\setmonofont[StylisticSet={1,3},Scale=.9]{inconsolata}
\RequirePackage{newtxmath}
  <% } else if (texlive === "tl2019") { -%>
<%# no special configuration required for lualatex to use the LaTeX default font -%>
  <% } else { -%>
% Source: https://tug.org/FontCatalogue/newcomputermoderntypewriter/

\setmainfont[%
ItalicFont=NewCM10-Italic.otf,%
BoldFont=NewCM10-Bold.otf,%
BoldItalicFont=NewCM10-BoldItalic.otf,%
SmallCapsFeatures={Numbers=OldStyle}]{NewCM10-Regular.otf}

\setsansfont[%
ItalicFont=NewCMSans10-Oblique.otf,%
BoldFont=NewCMSans10-Bold.otf,%
BoldItalicFont=NewCMSans10-BoldOblique.otf,%
SmallCapsFeatures={Numbers=OldStyle}]{NewCMSans10-Regular.otf}

\setmonofont[ItalicFont=NewCMMono10-Italic.otf,%
BoldFont=NewCMMono10-Bold.otf,%
BoldItalicFont=NewCMMono10-BoldOblique.otf,%
SmallCapsFeatures={Numbers=OldStyle}]{NewCMMono10-Regular.otf}

\setmathfont{NewCMMath-Regular.otf}
  <% } -%>
<% } else { -%>
<%# pdflatex -%>
  <% if (font == "arial") { -%>
% EN: Headings are typset in Helvetica (which is similar to Arial)
% DE: Schriftart fuer die Ueberschriften - ueberschreibt lmodern
\usepackage[scaled=.95]{helvet}
  <% } else if (font == "times") { -%>
\RequirePackage{newtxtext}
\RequirePackage{newtxmath}
\RequirePackage[zerostyle=b,scaled=.9]{newtxtt}
% EN: looks good with times, but no equivalent for lualatex found,
%     therefore could/should replaced with inconsolata
%\RequirePackage[zerostyle=b,scaled=.9]{newtxtt}
\RequirePackage[varl,scaled=.9]{inconsolata}
  <% } else { -%>
% This is the modern package for "Computer Modern".
% In case this gets activated, one has to switch from cmap package to glyphtounicode (in the case of pdflatex)
\usepackage[%
    rm={oldstyle=false,proportional=true},%
    sf={oldstyle=false,proportional=true},%
    % By using 'variable=true' the monospaced font can be used as variable font (with differents widths per letter)
    % However, this makes listings look ugly.
    tt={oldstyle=false,proportional=true,variable=false},%
    qt=false%
]{cfr-lm}
<% } -%>
% Has to be loaded AFTER any font packages. See https://tex.stackexchange.com/a/2869/9075.
\usepackage[T1]{fontenc}
<% } -%>

