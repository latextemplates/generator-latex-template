%% !!! If you change the font, be sure that words such as "workflow" can
%% !!! still be copied from the PDF. If this is not the case, you have
%% !!! to use glyphtounicode. See comment at cmap package.
%%
%% Background: "workflow" contains "fl" which is a ligature, which in turn
%%             is rendered as one character in the PDF and needs to be split
%%             whily copying.

<% if (documentclass == 'ustutt') { -%>
%% Clashes with mathdesign
% \PreventPackageFromLoading{amssymb,amsfont}
<%# inspired by https://github.com/programming-journal/programming/blob/main/programming.cls -%>
% Mathdesign spuriously redefines things it should not.
%
\makeatletter
\usepackage{etoolbox}
%
\csedef{P@MDback@fboxrule}{\the\fboxrule}
\csedef{P@MDback@fboxsep}{\the\fboxsep}
\csedef{P@MDback@arrayrulewidth}{\the\arrayrulewidth}
\cslet{P@MDback@bfdefault}{\bfdefault}
\cslet{P@MDback@footnoterule}{\footnoterule}
\cslet{P@MDback@hrulefill}{\hrulefill}
%
% "expert" option required commercial font - see https://tex.stackexchange.com/a/47479/9075
\RequirePackage[charter]{mathdesign}
%
\setlength{\fboxrule}{\csuse{P@MDback@fboxrule}}
\setlength{\fboxsep}{\csuse{P@MDback@fboxsep}}
\setlength{\arrayrulewidth}{\csuse{P@MDback@arrayrulewidth}}
\letcs{\bfdefault}{P@MDback@bfdefault}
\letcs{\footnoterule}{P@MDback@footnoterule}
\letcs{\hrulefill}{P@MDback@hrulefill}
% cleanup
\csundef{P@MDback@fboxrule}
\csundef{P@MDback@fboxsep}
\csundef{P@MDback@arrayrulewidth}
\csundef{P@MDback@bfdefault}
\csundef{P@MDback@footnoterule}
\csundef{P@MDback@hrulefill}
%
\makeatother

<% } -%>
<% if ((latexcompiler == "lualatex") || (latexcompiler == "both")) { indent = '' -%>
<% if (latexcompiler == "both") { indent = '  ' -%>
\ifluatex
<% } -%>
<%#
###########################################################################
lualatex
###########################################################################
-%>
<%= indent %>\usepackage[no-math]{fontspec}
<%= indent %>\usepackage{unicode-math}
<% switch (font) { case "arial": -%>
<%# Arial and LuaLaTeX -%>
<%# -%>

<%= indent %>\usepackage{ifplatform}
<%= indent %>\ifwindows
<%= indent %>  \setmainfont[Ligatures=TeX]{Arial}
<%= indent %>  \setsansfont[Ligatures=TeX]{Arial}
<%= indent %>\else
<%= indent %>  \usepackage{helvet} % source: comment to https://tex.stackexchange.com/q/212357/9075
<%= indent %>\fi
<%= indent %>% Use Inconsolata font (https://ctan.org/pkg/inconsolata)
<%= indent %>% See https://tex.stackexchange.com/q/2241/9075 for a list of other alternative mono-spaced fonts
<%= indent %>% No [Ligatures=TeX], because we want " rendered as upquotes, not as typographic correct ones
<%= indent %>% shapely l, upright quotes
<%= indent %>% If it is too large, add ",Scale=.9"
<%= indent %>\setmonofont[StylisticSet={1,3}]{Inconsolatazi4}
<% break; case "times": -%>
<%# Times New Roman and LuaLaTeX -%>
<%# -%>

<%= indent %>% See https://tug.org/FontCatalogue/texgyretermes/ for more information
<%= indent %>\setmainfont{texgyretermes}[
<%= indent %>  Extension = .otf,
<%= indent %>  UprightFont = *-regular,
<%= indent %>  BoldFont = *-bold,
<%= indent %>  ItalicFont = *-italic,
<%= indent %>  BoldItalicFont = *-bolditalic,
<%= indent %>  Ligatures=TeX
<%= indent %>]
<%= indent %>% See https://tug.org/FontCatalogue/texgyreheros/ for more information
<%= indent %>\setsansfont[Scale=.9]{TeX Gyre Heros Regular}
<% if (documentclass == 'lncs') { -%>
<%= indent %>% newtxtt looks good with times, but no equivalent for lualatex found,
<%= indent %>% therefore tried to replace with inconsolata.
<%= indent %>% However, inconsolata does not look good in the context of LNCS ...
<%= indent %>%\setmonofont[StylisticSet={1,3},Scale=.9]{inconsolata}
<%= indent %>% ... thus, we use the good old Latin Modern Mono font for source code.
<%= indent %>\setmonofont{Latin Modern Mono} % "variable=false"
<%# \setmonofont{Latin Modern Mono Prop} % "variable=true" -%>
<% } else { -%>
<%= indent %>% shapely l, upright quotes
<%= indent %>% Normal scaling is too large --> thus, we use ",Scale=.9"
<%= indent %>\setmonofont[StylisticSet={1,3},Scale=.9]{Inconsolatazi4}
<% } -%>
<%  break; default: -%>
<%# "default" font configuration -%>
<% switch (documentclass) { case "ieee": -%>
<%# Default font configuration of the IEEE template is used -%>
<% break; case "ustutt": -%>
<%# see also https://github.com/programming-journal/programming/blob/main/programming.cls -%>

<%= indent %>\usepackage{ifplatform}
<%= indent %>\ifwindows
<%= indent %>  \setmainfont{XCharter}
<%= indent %>  \setsansfont[Ligatures=TeX]{Arial}
<%= indent %>\else
<%= indent %>  \usepackage{helvet} % source: comment to https://tex.stackexchange.com/q/212357/9075
<%= indent %>  \setmainfont{XCharter}
<%= indent %>\fi

<%= indent %>% Use Inconsolata font (https://ctan.org/pkg/inconsolata)
<%= indent %>% See https://tex.stackexchange.com/q/2241/9075 for a list of other alternative mono-spaced fonts
<%= indent %>% No [Ligatures=TeX], because we want " rendered as upquotes, not as typographic correct ones
<%= indent %>% shapely l, upright quotes
<%= indent %>% If it is too large, add ",Scale=.9"
<%= indent %>\setmonofont[StylisticSet={1,3}]{Inconsolatazi4}
<% break; case "scientific-thesis": -%>

<%= indent %>% TODO
<% break; case "acmart": -%>
<%# no special configuration required -%>
<% break; default: -%>
<%# Default LaTeX Font and LuaLaTeX -%>
<% if (texlive >= 2021) { -%>

<%= indent %>% Typewriter font (for source code etc)
<%= indent %>% Use New Computer Modern font (Computer Modern is the default LaTeX font; this is the implemented modern variant)
<%= indent %>% Source: https://tug.org/FontCatalogue/newcomputermoderntypewriter/

<%= indent %>\setmainfont[
<%= indent %>  ItalicFont=NewCM10-Italic.otf,
<%= indent %>  BoldFont=NewCM10-Bold.otf,
<%= indent %>  BoldItalicFont=NewCM10-BoldItalic.otf,
<%= indent %>  SmallCapsFeatures={Numbers=OldStyle}]{NewCM10-Regular.otf}

<%= indent %>\setsansfont[
<%= indent %>  ItalicFont=NewCMSans10-Oblique.otf,
<%= indent %>  BoldFont=NewCMSans10-Bold.otf,
<%= indent %>  BoldItalicFont=NewCMSans10-BoldOblique.otf,
<%= indent %>  SmallCapsFeatures={Numbers=OldStyle}]{NewCMSans10-Regular.otf}

<%= indent %>\setmonofont[ItalicFont=NewCMMono10-Italic.otf,
<%= indent %>  BoldFont=NewCMMono10-Bold.otf,
<%= indent %>  BoldItalicFont=NewCMMono10-BoldOblique.otf,
<%= indent %>  SmallCapsFeatures={Numbers=OldStyle}]{NewCMMono10-Regular.otf}

<%= indent %>\setmathfont{NewCMMath-Regular.otf}
<% break; } -%>
<% } break; } -%>

<%= indent %>% Enable proper ligatures
<%= indent %>% For more information see https://ctan.org/pkg/selnolig
<%= indent %>% language "english" or "ngerman" is passed to selnolig by the document class
<%= indent %>\usepackage{selnolig}
<% } -%>
<% if ((latexcompiler != "lualatex") || (latexcompiler == "both")) { -%>
<% if (latexcompiler == "both") { -%>
\else
<% } -%>
<%#
###########################################################################
pdflatex
###########################################################################
-%>
<% if (latexcompiler == "both") { indent = '  ' } else { indent = '' } -%>
<% switch (font) { case "arial": -%>
<%# Arial and pdflatex -%>
<%# -%>
<%= indent %>% Headings are typeset in Helvetica (which is similar to Arial)
<%= indent %>\usepackage[scaled=.95]{helvet}
<% break; case "times": -%>
<%# Times New Roman and pdflatex -%>
<%# -%>
<%= indent %>\RequirePackage{newtxtext}
<%= indent %>\RequirePackage{newtxmath}
<%= indent %>\RequirePackage[zerostyle=b,scaled=.9]{newtxtt}
<% break; default: -%>
<%# "default" font configuration / Computer Modern and pdflatex -%>
<% switch (documentclass) { case "ieee": -%>
<%# The default font configuration (Palatino/Palladio) of the IEEE template is used.
In addition, we make the code appear more nice -%>
<%= indent %>% use nicer font for code
<%= indent %>\usepackage[zerostyle=b,scaled=.75]{newtxtt}
<% break; case "ustutt": -%>
<%# See also https://github.com/programming-journal/programming/blob/main/programming.cls -%>
<%= indent %>\usepackage{XCharter}
<%= indent %>\usepackage[scaled=.95]{helvet}
<% break; case "scientific-thesis": -%>
<%= indent %>% TODO
<% break; case "acmart": -%>
<%# no special configuration required -%>
<% break; default: -%>
<%= indent %>% This is the modern package for "Computer Modern".
<%= indent %>% In case this gets activated, one has to switch from cmap package to glyphtounicode (in the case of pdflatex)
<%= indent %>\usepackage[%
<%= indent %>  rm={oldstyle=false,proportional=true},%
<%= indent %>  sf={oldstyle=false,proportional=true},%
<%= indent %>  % By using 'variable=true' the monospaced font can be used as variable font (with differents widths per letter)
<%= indent %>  % However, this makes listings look ugly.
<%= indent %>  tt={oldstyle=false,proportional=true,variable=false},%
<%= indent %>  qt=false%
<%= indent %>]{cfr-lm}
<% } -%>
<% break; } -%>

<%= indent %>% Has to be loaded AFTER any font packages. See https://tex.stackexchange.com/a/2869/9075.
<%= indent %>\usepackage[T1]{fontenc}
<% if (latexcompiler == "both") { -%>
\fi
<% } -%>
<% } -%>
<% if (isThesis) { -%>

% DE: Noch mehr Symbole
%\usepackage{stmaryrd} %fuer \ovee, \owedge, \otimes
%\usepackage{marvosym} %fuer \Writinghand %patched to not redefine \Rightarrow
%\usepackage{mathrsfs} %mittels \mathscr{} schoenen geschwungenen Buchstaben erzeugen
%\usepackage{calrsfs} %\mathcal{} ein bisserl dickeren buchstaben erzeugen - sieht net so gut aus.

% EN: Fallback font - if the subsequent font packages do not define a font (e.g., monospaced)
%     This is the modern package for "Computer Modern".
%     In case this gets activated, one has to switch from cmap package to glyphtounicode (in the case of pdflatex)
% DE: Fallback-Schriftart
%\usepackage[%
%    rm={oldstyle=false,proportional=true},%
%    sf={oldstyle=false,proportional=true},%
%    tt={oldstyle=false,proportional=true,variable=true},%
%    qt=false%
%]{cfr-lm}

% EN: Headings are typeset in Helvetica (which is similar to Arial)
% DE: Schriftart fuer die Ueberschriften - ueberschreibt lmodern
%\usepackage[scaled=.95]{helvet}

% DE: Für Schreibschrift würde tun, muss aber nicht
%\usepackage{mathrsfs} %  \mathscr{ABC}

% EN: Font for the main text
% DE: Schriftart fuer den Fliesstext - ueberschreibt lmodern
%     Linux Libertine, siehe http://www.linuxlibertine.org/
%     Packageparamter [osf] = Minuskel-Ziffern
%     rm = libertine im Brottext, Linux Biolinum NICHT als serifenlose Schrift, sondern helvet (von oben) beibehalten
%\usepackage[rm]{libertine}

% EN: Alternative Font: Palantino. It is recommeded by Prof. Ludewig for German texts
% DE: Alternative Schriftart: Palantino, Packageparamter [osf] = Minuskel-Ziffern
%     Bitte nur in deutschen Texten
%\usepackage{mathpazo} %ftp://ftp.dante.de/tex-archive/fonts/mathpazo/ - Tipp aus DE-TEX-FAQ 8.2.1
% EN: The euro sign
% DE: Das Euro Zeichen
%     Fuer Palatino (mathpazo.sty): richtiges Euro-Zeichen
%     Alternative: \usepackage{eurosym}
% \newcommand{\EUR}{\ppleuro}

% DE: Schriftart fuer Programmcode - ueberschreibt lmodern
%     Falls auskommentiert, wird die Standardschriftart lmodern genommen
%     Fuer schreibmaschinenartige Schluesselwoerter in den Listings - geht bei alten Installationen nicht, da einige Fontshapes (<>=) fehlen
%\usepackage[scaled=.92]{luximono}
%\usepackage{courier}
% DE: BeraMono als Typewriter-Schrift, Tipp von http://tex.stackexchange.com/a/71346/9075
%\usepackage[scaled=0.83]{beramono}

\usepackage{setspace}
% Alternative package: https://ctan.org/pkg/leading

% Symbole Check und Cross
\usepackage{pifont}
\newcommand{\dingcheck}{\ding{51}}
\newcommand{\dingcross}{\ding{55}}
%for scaling see http://tex.stackexchange.com/a/130236/9075

% DE: Noch mehr Symbole
%\usepackage{stmaryrd} %fuer \ovee, \owedge, \otimes
%\usepackage{marvosym} %fuer \Writinghand %patched to not redefine \Rightarrow
%\usepackage{mathrsfs} %mittels \mathscr{} schoenen geschwungenen Buchstaben erzeugen
%\usepackage{calrsfs} %\mathcal{} ein bisserl dickeren buchstaben erzeugen - sieht net so gut aus.
<% } -%>
<% if (documentclass == 'scientific-thesis') { -%>

\usepackage[automark]{scrlayer-scrpage}
\automark[section]{chapter}
\setkomafont{pageheadfoot}{\normalfont\sffamily}
\setkomafont{pagenumber}{\normalfont\sffamily}

\ihead[]{}
\chead[]{}
\ohead[]{\headmark}
\cfoot[]{}
\ofoot[\usekomafont{pagenumber}\thepage]{\usekomafont{pagenumber}\thepage}
\ifoot[]{}
<% } -%>
<% if (documentclass == 'ustutt') { -%>

% Zusatzliche Symbole (Text Companion font extension)
\usepackage{textcomp}

\setkomafont{sectioning}{\normalfont\normalcolor\rmfamily\mdseries}
\setkomafont{descriptionlabel}{\bfseries}
\setkomafont{pageheadfoot}{\small\sffamily}
\setkomafont{pagenumber}{\small\sffamily\bfseries}

\typearea[current]{last}
<% } -%>
