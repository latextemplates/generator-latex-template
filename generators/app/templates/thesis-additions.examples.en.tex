<%= heading2 %>{Miscellaneous Examles}
\label{ssec:example}

Referencetest: \Cref{ssec:example}, \cref{fig:Abbildung} und \cref{alg:example}.

<%- bexample %>
Checkmark: \dingcheck.
Crossmark: \dingcross.
<%- eexample %>

\begin{figure}
  \missingfigure{}
  \caption{Abbildung}
  \label{fig:Abbildung}
\end{figure}

\begin{landscape}
  \begin{figure}
    \missingfigure{}
    \caption{Gedrehte Abbildung}
    \label{fig:AbbildungGedreht}
  \end{figure}
\end{landscape}

<%= heading3 %>{Algorithmen}

\begin{algorithm}
  \caption{$algo$}
  \label{alg:example}
  \begin{algorithmic}[1]
    \State $a \gets 0$
    \State State 2\label{alg1:state2}
  \end{algorithmic}
\end{algorithm}

\begin{algorithm}
  \caption{Algorithmus 2}
  \label{alg:example2}
  \begin{algorithmic}[1]
    \State $a \gets 0$
    \State State 2\label{alg2:state2}
  \end{algorithmic}
\end{algorithm}

\Cref{alg:example} hat bereits einen Algorithmus gezeigt.
Test der Zeilenreferenzierung: Zeile~\ref{alg1:state2} (\cref{alg:example}) und Zeile~\ref{alg2:state2} (\cref{alg:example2}).

<%= heading3 %>{Definitionen}
\begin{definition}[Title]
  \label{def:def1}
  Definition Text
\end{definition}

\Cref{def:def1} zeigt \ldots

<%= heading3 %>{Aufzählungen}

\begin{enumerate}[label=\alph*)]
  \item a
  \item b
  \item c
  \item d
\end{enumerate}

Equivalent to paralist's inparaenum:
\begin{enumerate*}[label=\alph*)]
  \item a
  \item b
  \item c
  \item d
\end{enumerate*}

\begin{description}
  \item[first] Erstens
  \item[second] Zweitens
  \item[third] Drittens
\end{description}

\begin{description}
  \item[\texttt{first}] Erstens
  \item[\texttt{second}] Zweitens
  \item[\texttt{third}] Drittens
\end{description}

%works only if package enumitem is loaded
\begin{description}[font=\ttfamily]
  \item[first] Erstens
  \item[second] Zweitens
  \item[third] Drittens
\end{description}

\begin{description}[style=unboxed]
  \item[first label with a long description text breaking over one line. Enabled by enumitem package] Erstens
  \item[second] Zweitens
  \item[third] Drittens
\end{description}

\begin{Description}
  \item[first label with a long description text breaking over one line. Defined in template.tex] Erstens
  \item[second] Zweitens
  \item[third] Drittens
\end{Description}

\begin{itemize}
  \item Erstens
  \item Zweitens
  \item Drittens
\end{itemize}

Optionaler Parameter ändert den Marker, der vorangestellt ist.
Siehe \url{http://www.weinelt.de/latex/item.html}.
\begin{itemize}
  \item[A] Erstens
  \item[B] Zweitens
  \item[C] Drittens
\end{itemize}

Falsche Benutzung des optionalen Parameters wie folgt:
\begin{itemize}
  \item[first] Erstens
  \item[second] Zweitens
  \item[third] Drittens
\end{itemize}
Dabei ist zu beachten, dass es sich bei Einbindung von \texttt{enumitem} anders verhält als bei \texttt{paralist}.

<%= heading3 %>{fquote}

\begin{fquote}[T.\ Informatiker]
  Bis nächsten Freitag ist das Programm fertig.
\end{fquote}

\begin{gfquote}{T.\ Informatiker}
  Bis nächsten Freitag ist das Programm fertig.
\end{gfquote}
