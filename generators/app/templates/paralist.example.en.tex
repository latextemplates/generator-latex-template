<%= heading2 %>{Itemization}

One can list items as follows:

<%- bexample %>
\begin{itemize}
  \item Item One
  \item Item Two
\end{itemize}
<%- eexample %>

<% if (documentclass != "lncs") { -%>
With the package paralist, one can create itemizations with lesser spacing:

<%- bexample %>
\begin{compactitem}
  \item Item One
  \item Item Two
\end{compactitem}
<%- eexample %>
<% } -%>

One can enumerate items as follows:

<%- bexample %>
\begin{enumerate}
  \item Item One
  \item Item Two
\end{enumerate}
<%- eexample %>

<% if (documentclass != "lncs") { -%>
With the package paralist, one can create enumerations with lesser spacing:

<%- bexample %>
\begin{compactenum}
  \item Item One
  \item Item Two
\end{compactenum}
<%- eexample %>
<% } -%>

With paralist, one can even have all items typeset after each other and have them clean in the TeX document:

<%- bexample %>
\begin{inparaenum}
  \item All these items...
  \item ...appear in one line
  \item This is enabled by the paralist package.
\end{inparaenum}
<%- eexample %>
