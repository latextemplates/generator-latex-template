% !TeX spellcheck = de-DE
% !TeX encoding = utf8
% !TeX program = <%= latexcompiler %>
% !BIB program = <%= bibtextool %>
% -*- coding:utf-8 mod:LaTeX -*-

<% if (documentclass === 'scientific-thesis') { %>
% The following package allows \\ at the title page
% For more information see https://github.com/latextemplates/scientific-thesis-cover/issues/4
\RequirePackage{kvoptions-patch}
<% } %>

<% if (latexcompiler !== 'lualatex') { %>
% Neue deutsche Trennmuster
% Siehe http://www.ctan.org/pkg/dehyph-exptl und http://projekte.dante.de/Trennmuster/WebHome
% Nur für pdflatex, nicht für lualatex
\RequirePackage[ngerman=ngerman-x-latest]{hyphsubst}
<% } %>

\documentclass[
  % fontsize=11pt is the standard
  % ()Aus scrguide.pdf - der Dokumentation von KOMA-Script)
  % Nach DUDEN steht in Gliederungen, in denen ausschließlich arabische Ziffern für die Nummerierung
  % verwendet werden, am Ende der Gliederungsnummern kein abschließender Punkt
  % (siehe [DUD96, R3]). Wird hingegen innerhalb der Gliederung auch mit römischen Zahlen
  % oder Groß- oder Kleinbuchstaben gearbeitet, so steht am Ende aller Gliederungsnummern ein
  % abschließender Punkt (siehe [DUD96, R4])
  numbers=autoendperiod,
  a4paper,  % Standard format - only KOMAScript uses paper=a4 - https://tex.stackexchange.com/a/61044/9075
  twoside,  % we are optimizing for both screen and two-side printing. So the page numbers will jump, but the content is configured to stay in the middle (by using the geometry package)
  bibliography=totoc,
  %               idxtotoc,   % Index ins Inhaltsverzeichnis
  %               liststotoc, % List of X ins Inhaltsverzeichnis, mit liststotocnumbered werden die Abbildungsverzeichnisse nummeriert
  headsepline,
  cleardoublepage=empty,
  parskip=half,
  %               draft    % um zu sehen, wo noch nachgebessert werden muss - wichtig, da Bindungskorrektur mit drin
  draft=false
]{scrbook}

<%- include('microtype.de.preamble.tex', this); %>
<%- include('babel.preamble.tex', this); %>
<%- include('url.de.preamble.tex', this); %>
<%- include('hyperref.en.preamble.tex', this); %>
<% if (cleveref) { %><%- include('cleveref.de.preamble.tex', this); %><% } %>

\begin{document}

<%= heading1 %>{LaTeX Hinweise}
\label{chap:latexhints}

Hier sollen allgemeine \LaTeX-Hinweise gegeben werden, damit man Minimalbeispiele vorliegen hat, um sofort loszulegen.

<%= heading2 %>{Trennung von Absätzen}

Pro Satz eine neue Zeile.
Das ist wichtig, um sauber versionieren zu können.
In LaTeX werden Absätze durch eine Leerzeile getrennt.
Analogie zu Word: Bei Word werden neue Absätze durch einmal Eingabetaste gemacht.
Dies führt bei LaTeX jedoch nicht zu einem neuen Absatz, da LaTeX direkt aufeinanderfolgende Zeilen zu einer Zeile zusammenfügt.
Möchte man nun einen Absatz haben, muss man zweimal die Eingabetaste drücken.
Dies führt zu einer leeren Zeile.
In Word gibt es die Funktion Großschreibetaste und Eingabetaste gleichzeitig.
Wenn man dies drückt, wird einer harter Umbruch erzwungen.
Der Text fängt am Anfang der neuen Zeile an.
In LaTeX erreicht man dies durch Doppelbackslashes (\textbackslash\textbackslash) erzeugt.
Dies verwendet man quasi nie.

Folglich werden neue Abstäze insbesondere \emph{nicht} durch Doppelbackslashes erzeugt.
Beispielsweise begann der letzte Satz in einem neuen Absatz.
Eine ausführliche Motivation hierfür findet sich in \url{http://loopspace.mathforge.org/HowDidIDoThat/TeX/VCS/#section.3}.

Möchte man die Art des Absatzes ändern, so kann man die Dokumentklassenoption \texttt{parskip} verwenden.
Beispielsweise kann man mit \texttt{parskip=off} erreichen, dass statt eines freien Bereichs die erste Zeile des Absatzes eingezogen wird.

<%- include('figure.de.example.tex', this); %>

\end{document}
