<% if (preparereitzig) { -%>
\RequirePackage{snapshot}
<% } -%>
<% switch (documentclass) { case "lncs": -%>
% Dieses Template wurde mit der "LLNCS DOCUMENT CLASS -- version 2.21 (12-Jan-2022)" getestet

<% break; case "ieee": -%>
% This template has been tested with IEEEtran of 2015.

<% break; } -%>
% !TeX spellcheck = de-DE
% LTeX: language=de-DE
% !TeX encoding = utf8
% !TeX program = <%= latexcompiler %>
<% if (requiresShellEscape) { -%>
% !TeX TXS-program:compile = txs:///<%= latexcompiler %>/[--shell-escape]
<% } -%>
% !BIB program = <%= bibtextool %>
% -*- coding:utf-8 mod:LaTeX -*-
<% if (latexcompiler !== 'lualatex') { -%>
% Neue deutsche Trennmuster
% Siehe http://www.ctan.org/pkg/dehyph-exptl und http://projekte.dante.de/Trennmuster/WebHome
% Nur für pdflatex, nicht für lualatex
\RequirePackage[ngerman=ngerman-x-latest]{hyphsubst}
<% } -%>
<% switch (documentclass) { case "lncs": -%>

% "a4paper" enables:
%
%  - easy print out on DIN A4 paper size
%
% One can configure a4 vs. letter in the LaTeX installation. So it is configuration dependend, what the paper size will be.
% This option  present, because the current word template offered by Springer is DIN A4.
% We accept that DIN A4 cause WTFs at persons not used to A4 in USA.
%
% "runningheads":
%
%  - zeigt Author + Titel auf jeder Seite.
%  - Während des Schreibens und das Review des Papers hilft das, um z.B. auf konkrete Seitenzahlen einfach verweisen zu können.
%
% This is good for other readers to enable proper archiving among other papers and pointing to
% content. Even if the title page states the title, when printed and stored in a folder, when
% blindly opening the folder, one could hit not the title page, but an arbitrary page. Therefore,
% it is good to have title printed on the pages, too.
%
% Diese Option nach Aufforderung durch die Herausgeber entfernen.
%
% To disable outputting page headers and footers, remove "runningheads"
\documentclass[runningheads,<%= papersize %>paper,ngerman]{llncs}[2022/01/12]
<% break; case "acmart": -%>

<% if (texlive >= 2021) { -%>
% We balance the last page using the package "pbalance" (see https://ctan.org/pkg/pbalance), because it "just works" (in contrast to balance.sty or other solutions)
<% } -%>
\documentclass[<%= acm_format %>,<% if (acm_review) { %>review,<% } %><%= papersize %>paper,<% if (texlive >= 2021) { %>pbalance<% } else { %>balance<% } %>]{acmart}
<% break; case "ieee": -%>

% DO NOT DOWNLOAD IEEEtran.cls - Use the one of your LaTeX distribution
% For the final version, replace "draftcls" by "final"
\documentclass[<%= ieee_variant %>,<%= papersize %>paper]{IEEEtran}[2015/08/26]
<% if (texlive >= 2021) { -%>

% Balance the last page
% The pbalance package (see https://ctan.org/pkg/pbalance) "just works" (in contrast to balance.sty or other solutions)
\usepackage{pbalance}
<% } else { -%>

% Balance the last page using the balance package (see https://ctan.org/pkg/balance)
\usepackage{balance}
<% } -%>
<% break; case "scientific-thesis": -%>

% The following package allows \\ at the title page
% For more information see https://github.com/latextemplates/scientific-thesis-cover/issues/4
\RequirePackage{kvoptions-patch}
\documentclass[
  % fontsize=11pt is the standard
  % ()Aus scrguide.pdf - der Dokumentation von KOMA-Script)
  % Nach DUDEN steht in Gliederungen, in denen ausschließlich arabische Ziffern für die Nummerierung
  % verwendet werden, am Ende der Gliederungsnummern kein abschließender Punkt
  % (siehe [DUD96, R3]). Wird hingegen innerhalb der Gliederung auch mit römischen Zahlen
  % oder Groß- oder Kleinbuchstaben gearbeitet, so steht am Ende aller Gliederungsnummern ein
  % abschließender Punkt (siehe [DUD96, R4])
  numbers=autoendperiod,
  ngerman,  % Neue deutsche Rechtschreibung; der Parameter wird an andere Pakete weiter gegeben
  a4paper,  % Standard format - only KOMAScript uses paper=a4 - https://tex.stackexchange.com/a/61044/9075
  twoside,  % we are optimizing for both screen and two-side printing. So the page numbers will jump, but the content is configured to stay in the middle (by using the geometry package)
  bibliography=totoc,
  % idxtotoc,   % Index ins Inhaltsverzeichnis
  % liststotoc, % List of * ins Inhaltsverzeichnis, mit liststotocnumbered werden die Abbildungsverzeichnisse nummeriert
  headsepline,
  cleardoublepage=empty,
  parskip=half,
  %               draft    % um zu sehen, wo noch nachgebessert werden muss - wichtig, da Bindungskorrektur mit drin
  draft=false
]{scrbook}
\usepackage{scrlayer-scrpage}
\ihead[]{}
\chead[]{}
\ohead[]{\headmark}
\cfoot[]{}
\ofoot[\usekomafont{pagenumber}\thepage]{\usekomafont{pagenumber}\thepage}
\ifoot[]{}
<% break; default: -%>

% not yet supported template. Please report to the author of generator-latex-template
\documentclass{scrbook}
<% } -%>

% backticks (`) werden als solches in verbatim-Umgebungen dargestellt
% Details unter:
%   - https://tex.stackexchange.com/a/341057/9075
%   - https://tex.stackexchange.com/a/47451/9075
%   - https://tex.stackexchange.com/a/166791/9075
\usepackage{upquote}

<%- include('babel.preamble.de.tex', this); %>
<%- include('url.preamble.de.tex', this); %>
<%- include('font.preamble.en.tex', this); %>
<%- include('microtype.preamble.de.tex', this); %>
\usepackage{graphicx}

<%- include('diagbox.preamble.en.tex', this); -%>

<%# Required for package pdfcomment later -%>
\usepackage{xcolor}

<% if (listings == 'listings') { -%>
<%- include('listings.preamble.de.tex', this); %>
<% } else { -%>
<%- include('minted.preamble.en.tex', this); %>
<% } -%>

<% if (enquotes == 'csquotes')  { %><%- include('csquotes.preamble.en.tex', this); %><% } -%>
<% if (enquotes == 'textcmds')  { %><%- include('textcmds.preamble.en.tex', this); %><% } -%>

<%- include('booktabs.preamble.en.tex', this); %>
<%- include('paralist.preamble.en.tex', this); %>
<% if (documentclass == 'lncs')  { -%>
<%- include('floatflt.preamble.en.tex', this); %>
% Bibliopgraphy enhancements
%  - enable \cite[prenote][]{ref}
%  - enable \cite{ref1,ref2}
% Alternative: \usepackage{cite}, which enables \cite{ref1, ref2} only (otherwise: Error message: "White space in argument")

% Doc: http://texdoc.net/natbib
\usepackage[%
  square,        % for square brackets
  comma,         % use commas as separators
  numbers,       % for numerical citations;
%  sort,          % orders multiple citations into the sequence in which they appear in the list of references;
  sort&compress, % as sort but in addition multiple numerical citations
                 % are compressed if possible (as 3-6, 15);
]{natbib}
% In the bibliography, references have to be formatted as 1., 2., ... not [1], [2], ...
\renewcommand{\bibnumfmt}[1]{#1.}

% Prepare more space-saving rendering of the bibliography
% Source: https://tex.stackexchange.com/a/280936/9075
\SetExpansion
[ context = sloppy,
  stretch = 30,
  shrink = 60,
  step = 5 ]
{ encoding = {OT1,T1,TS1} }
{ }

<% } -%>
<% if (todo == 'pdfcomment') { -%>

<%- include('pdfcomment.preamble.en.tex', this); -%>
<% } else { -%>

<%- include('plainlatextodo.preamble.en.tex', this); -%>
<% } -%>

% Fußnoten unter Gleitumgebungen ("floats") platzieren
% Quelle: https://tex.stackexchange.com/a/32993/9075
\usepackage{stfloats}
\fnbelowfloat

<%- include('siunitx.preamble.en.tex'); %>
<%- include('hyperref.preamble.de.tex', this); %>
<% if (texlive >= 2021) { %><%- include('mindflow.preamble.en.tex', this); %><% } -%>
<%- include('cleveref.preamble.de.tex', this); %>
\usepackage{lipsum}
<% if (examples) { -%>

% For demonstration purposes only
% These packages can be removed when all examples have been deleted
\usepackage[math]{blindtext}
\usepackage{mwe}
<% if (useExampleEnvironment) { -%>
\usepackage[realmainfile]{currfile}
\usepackage{tcolorbox}
\tcbuselibrary{<%= listings %>}
<% } -%>
<% } -%>

<%- include('powerset.preamble.en.tex', this); -%>

\usepackage{xspace}
%\newcommand{\eg}{e.\,g.\xspace}
%\newcommand{\ie}{i.\,e.\xspace}
\newcommand{\eg}{e.\,g.,\ }
\newcommand{\ie}{i.\,e.,\ }

% Enable hyphenation at other places as the dash.
% Example: applicaiton\hydash specific
\makeatletter
\newcommand{\hydash}{\penalty\@M-\hskip\z@skip}
% Definition of "= taken from http://mirror.ctan.org/macros/latex/contrib/babel-contrib/german/ngermanb.dtx
\makeatother

% correct bad hyphenation here
\hyphenation{op-tical net-works semi-conduc-tor}
<% switch (documentclass) { case "acmart": -%>

\acmConference[short-conference-name]{full-conference-name}{November 02--06, 2021}{City, Country}
\acmDOI{10.1145/9999997.9999999}
%\startPage{457}
<% break; case "lncs": -%>

% Add copyright
%
% This is recommended if you intend to send the version to colleagues
% See https://ctan.org/pkg/llncsconf for details
\iffalse
  % state: intended | submitted | llncs
  % you can add "crop" if the paper should be cropped to the format Springer is publishing
  \usepackage[intended]{llncsconf}

  \conference{name of the conference}

  % in case of "proceedings" (final version!)
  % example: \llncs{Anonymous et al. (eds). \emph{Proceedings of the International Conference on \LaTeX-Hacks}, LNCS~42. Some Publisher, 2016.}{0042}
  % 0042 denotes an example start page
  \llncs{book editors and title}{0042}
\fi
<% break; case "scientific-thesis": -%>

<% if (feature.acronyms) { -%><%- include('acronyms.preamble.en.tex', this); %><% } -%>
<%- include('scientific-thesis-cover.preamble.en.tex', this); %>
<% break; } -%>
<% if (latexcompiler == "pdflatex") { -%>

% Enable copy and paste of text from the PDF
% Only required for pdflatex. It "just works" in the case of lualatex.
% Alternative: cmap or mmap package
% mmap enables mathematical symbols, but does not work with the newtx font set
% See: https://tex.stackexchange.com/a/64457/9075
% Other solutions outlined at http://goemonx.blogspot.de/2012/01/pdflatex-ligaturen-und-copynpaste.html and http://tex.stackexchange.com/questions/4397/make-ligatures-in-linux-libertine-copyable-and-searchable
% Trouble shooting outlined at https://tex.stackexchange.com/a/100618/9075
%
% According to https://tex.stackexchange.com/q/451235/9075 this is the way to go
\input glyphtounicode
\pdfgentounicode=1
<% } -%>

\begin{document}
<% switch (documentclass) { case "lncs": -%>

\title{Paper Title}
% If Title is too long, use \titlerunning
%\titlerunning{Short Title}

% Single insitute
\author{Firstname Lastname \and Firstname Lastname}

% If there are too many authors, use \authorrunning
%\authorrunning{First Author et al.}

\institute{Institute}

%% Multiple insitutes - ALTERNATIVE to the above
% \author{%
%     Firstname Lastname\inst{1} \and
%     Firstname Lastname\inst{2}
% }
%
%If there are too many authors, use \authorrunning
%  \authorrunning{First Author et al.}
%
%  \institute{
%      Insitute 1\\
%      \email{...}\and
%      Insitute 2\\
%      \email{...}
%}

\maketitle

\begin{abstract}
  \lipsum[1]
\end{abstract}

\begin{keywords}
  keyword1, keyword2
\end{keywords}

\section{Introduction}
\label{sec:intro}
\lipsum[1-3]\todo{Refine me}

The remainder of the paper starts with a presentation of related work (\cref{sec:relatedwork}).
<% if (examples) { %>It is followed by a presentation of hints on \LaTeX{} (\cref{sec:latexhints}).
<% } -%>
<% if (!examples && (documentclass == 'ieee')) { %>Afterwards, we present some Lorem Ipsum (\cref{sec:loremipsum}).
<% } -%>
Finally, a conclusion is drawn and outlook on future work is made (\cref{sec:outlook}).

\section{Related Work}
\label{sec:relatedwork}

Winery~\cite{Winery} is a graphical \commentontext{modeling}{modeling with one ``l'', because of AE} tool.
The whole idea of TOSCA is explained by <% if (available.citet) { %>\citet<% } else { %>Binz et al.~\cite<% } %>{Binz2009}.
<%  break;
default:  -%>
\pagenumbering{arabic}
\Titelblatt

%Eigener Seitenstil fuer die Kurzfassung und das Inhaltsverzeichnis
\deftriplepagestyle{preamble}{}{}{}{}{}{\pagemark}
%Doku zu deftriplepagestyle: scrguide.pdf
\pagestyle{preamble}
\renewcommand*{\chapterpagestyle}{preamble}

%Kurzfassung / abstract
%auch im Stil vom Inhaltsverzeichnis
\section*{Kurzfassung}

<Kurzfassung der Arbeit>

\cleardoublepage

\microtypesetup{protrusion=false}

% In case you have trouble with headings reaching into the page numbers, enable the following three lines.
% Hint by http://golatex.de/inhaltsverzeichnis-schreibt-ueber-rand-t3106.html
%
%\makeatletter
%\renewcommand{\@pnumwidth}{2em}
%\makeatother
%
% Bei einem ungünstigen Seitenumbruch im Inhaltsverzeichnis, kann dieser mit
% \addtocontents{toc}{\protect\newpage}
% an der passenden Stelle im Fließtext erzwungen werden.
\tableofcontents

\listoffigures

\listoftables
<% if (listings == 'listings') { -%>

% We use lstlisting environments with caption paramters.
% Thus, we need that command.
% Alternative: \listof{Listing}{List of Listings}
\lstlistoflistings
<% } else { -%>

\listof{Listing}{Verzeichnis der Listings}
<% } -%>

%mittels \newfloat wurde die Algorithmus-Gleitumgebung definiert.
%Mit folgendem Befehl werden alle floats dieses Typs ausgegeben
%\listof{Algorithmus}{Verzeichnis der Algorithmen}
%\listofalgorithms %Ist nur für Algorithmen, die mittels \begin{algorithm} umschlossen werden, nötig

% Abkürzungsverzeichnis
\printnoidxglossaries

\microtypesetup{protrusion=true}

% Headline and footline
\renewcommand*{\chapterpagestyle}{scrplain}
\pagestyle{scrheadings}
\pagestyle{scrheadings}
\ihead[]{}
\chead[]{}
\ohead[]{\headmark}
\cfoot[]{}
\ofoot[\usekomafont{pagenumber}\thepage]{\usekomafont{pagenumber}\thepage}
\ifoot[]{}

\chapter{Einleitung}
In diesem Kapitel steht die Einleitung zu dieser Arbeit.
Sie soll nur als Beispiel dienen und hat nichts mit dem Buch \cite{WSPA} zu tun.
Nun viel Erfolg bei der Arbeit!

Bei \LaTeX\ werden Absätze durch freie Zeilen angegeben.
Da die Arbeit über ein Versionskontrollsystem versioniert wird, ist es sinnvoll, pro \emph{Satz} eine neue Zeile im \texttt{.tex}-Dokument anzufangen.
So kann einfacher ein Vergleich von Versionsständen vorgenommen werden.

Die Arbeit ist in folgender Weise gegliedert:
In \cref{chap:k2} werden die Grundlagen dieser Arbeit beschrieben.
Schließlich fasst \cref{chap:zusfas} die Ergebnisse der Arbeit zusammen und stellt Anknüpfungspunkte vor.


\chapter{Kapitel zwei}
\label{chap:k2}

Hier wird der Hauptteil stehen.
Falls mehrere Kapitel gewünscht, entweder mehrmals \texttt{\textbackslash{}chapter} benutzen oder pro Kapitel eine eigene Datei anlegen und \texttt{ausarbeitung.tex} anpassen.

LaTeX-Hinweise stehen in \cref{chap:latextipps}.

%noch etwas Fülltext
\blinddocument

\chapter{Verwandte Arbeiten}
Eine Beschreibung relevanter wissenschaftlicher Arbeiten mit Bezug zur eigenen Arbeit. Der Abschnitt kann je nach Kontext auch an anderer Stelle stehen.

\chapter{Zusammenfassung und Ausblick}\label{chap:zusfas}
Hier bitte einen kurzen Durchgang durch die Arbeit.

\section*{Ausblick}
...und anschließend einen Ausblick
<% break; } -%>

<% if (examples) { -%>
<%= heading1 %>{LaTeX Hinweise}
\label{chap:latexhints}

<% if (useExampleEnvironment) { -%>
% Benötigt für eine korrekte Darstellung der Hinweise im erzeugten PDF
\newcount\LTGbeginlineexample
\newcount\LTGendlineexample
\newenvironment{ltgexample}%
{\LTGbeginlineexample=\numexpr\inputlineno+1\relax}%
{\LTGendlineexample=\numexpr\inputlineno-1\relax%
%
\tcbinputlisting{%
  listing only,
  listing file=\currfilepath,
  colback=green!5!white,
  colframe=green!25,
  coltitle=black!90,
  coltext=black!90,
  left=8mm,
  title=Zugehöriger \LaTeX{}-Quelltext aus \texttt{\currfilepath},
<% if (listings == 'listings') { -%>
  listing options={%
    frame=none,
    language={[LaTeX]TeX},
    escapeinside={},
    firstline=\the\LTGbeginlineexample,
    lastline=\the\LTGendlineexample,
    firstnumber=\the\LTGbeginlineexample,
    basewidth=.5em,
    aboveskip=0mm,
    belowskip=0mm,
    numbers=left,
    xleftmargin=0mm,
    numberstyle=\tiny,
    numbersep=8pt%
  }
<% } else { -%>
    minted language=TeX,
    minted style=vs,
    minted options={
      fontsize=\footnotesize,
      firstline=\the\LTGbeginlineexample,
      lastline=\the\LTGendlineexample,
      firstnumber=\the\LTGbeginlineexample,
      breaklines,
      linenos,
      numbersep=8pt
    }
<% } -%>
}
}%
<% } -%>

Hier sollen allgemeine \LaTeX-Hinweise gegeben werden, damit man Minimalbeispiele vorliegen hat, um sofort loszulegen.

<%- include('paragraphs.example.de.tex', this); %>
<% if (texlive >= 2021) { %><%- include('mindflow.example.en.tex', this); %>
<% } -%>
<%- include('hyphenation.example.en.tex', this); %>
<%- include('siunitx.example.en.tex', this); %>
<% if (enquotes == 'csquotes') { %><%- include('csquotes.example.en.tex', this); %>
<% } -%>
<% if (enquotes == 'textcmds') { %><%- include('textcmds.example.en.tex', this); %>
<% } -%>
<% if (enquotes == 'plainlatex') { %><%- include('plainlatex.enquotes.example.de.tex', this); %>
<% } -%>
<%- include('cleveref.example.en.tex', this); %>
<%- include('figure.example.de.tex', this); %>
<%- include('subfloat.example.en.tex', this); %>
<%- include('tables.example.en.tex', this); %>
<% if (listings == 'listings') { -%>
<%- include('listings.example.de.tex', this); %>
<% } else { -%>
<%- include('minted.example.en.tex', this); %>
<% } -%>
<%- include('paralist.example.en.tex', this); %>
<% if (feature.acronyms) { %><%- include('acronyms.example.de.tex', this); %>
<% } -%>
<%- include('otherfeatures.example.en.tex', this); %>
<% } -%>
<% if (howtotext) { -%>

%%% ===============================================================================
<%= heading1 %>{Zusammenfassung und Ausblick}\label{sec:conclusion}
%%% ===============================================================================

Your conclusions are not just a factual summary of your work, but they position, interpret, and defend your findings against the state of the art that you discussed in \cref{sec:introduction:state-of-art}. You specifically outline which concrete findings or methodological contributions advance our knowledge towards the general objective you introduced in \cref{sec:introduction:topic}. Objectively discuss which parts you solved and in which parts you failed.

You should explicitly discuss limitations and shortcomings of your work and detail what kind of future studies are needed to overcome these limitations. Be specific in the sense that your arguments for future work should be based on concrete findings and insights you obtained in your report.
<% } else { -%>
<% if (!examples && ((documentclass == 'acmart') || (documentclass == 'ieee'))) { -%>

<%= heading1 %>{Lorem ipsum}
\label{sec:loremipsum}
\lipsum[1-4]
<% } -%>

<%= heading1 %>{Zusammenfassung und Ausblick}
\label{sec:outlook}
\lipsum[1-2]
<% } -%>

<% switch (documentclass) { case "lncs": -%>
\subsubsection*{Danksagungen}

<% break; case "acmart": -%>
\begin{acks}
<% break; case "ieee": -%>
% regular IEEE prefers the singular form
\section*{Danksagung}

<% break; default: -%>
<% break; } -%>
<% if ((documentclass == 'acmart') || (documentclass == 'lncs') || (documentclass == 'ieee')) { -%>
Identification of funding sources and other support, and thanks to individuals and groups that assisted in the research and the preparation of the work should be included in an acknowledgment section, which is placed just before the reference section in your document \cite{acmart}.
<% if (howtotext) { -%>

The hints on writing a paper/thesis are taken from Dirk Fahland's \href{https://github.com/dfahland/Master-or-Bachelor-thesis-Template-Eindhoven-University-of-Technology}{LaTeX template for Bachelor and Master theses at Eindhoven University of Technology}.
<% } -%>
<% if (documentclass == 'acmart') { -%>

For more \LaTeX{} hints for ACM read on at \url{https://www.acm.org/publications/taps/latex-best-practices}.
\end{acks}
<% } -%>
<% } -%>

%%% ===============================================================================
%%% Bibliography
%%% ===============================================================================

In the bibliography, use \texttt{\textbackslash textsuperscript} for <%- bquote %>st<%- equote %>, <%- bquote %>nd<%- equote %>, \ldots:
E.g., <%- bquote %>The 2\textsuperscript{nd} conference on examples<%- equote %>.
When you use \href{https://www.jabref.org}{JabRef}, you can use the clean up command to achieve that.
See \url{https://help.jabref.org/en/CleanupEntries} for an overview of the cleanup functionality.
<% switch (documentclass) { case "lncs": -%>

\renewcommand{\bibsection}{\section*{Literatur}} % requried for natbib to have "References" printed and as section*, not chapter*
% Use natbib compatbile splncs04nat style.
% It does provide all features of splncs03, but is developed in a clean way.
% Source: http://phaseportrait.blogspot.de/2011/02/natbib-compatible-bibtex-style-bst-file.html
\bibliographystyle{splncs04nat}
\begingroup
  \microtypecontext{expansion=sloppy}
  \small % ensure correct font size for the bibliography
  \bibliography{<%= filenames.bib %>}
\endgroup
<%  break; case "acmart": -%>

\bibliographystyle{ACM-Reference-Format}
\bibliography{<%= filenames.bib %>}
<%  break; case "ieee": -%>

% trigger a \newpage just before the given reference
% number - used to balance the columns on the last page
% adjust value as needed - may need to be readjusted if
% the document is modified later
%\IEEEtriggeratref{8}
% The "triggered" command can be changed if desired:
%\IEEEtriggercmd{\enlargethispage{-5in}}

% Enable to reduce spacing between bibitems (source: https://tex.stackexchange.com/a/25774)
% \def\IEEEbibitemsep{0pt plus .5pt}

\bibliographystyle{IEEEtranN} % IEEEtranN is the natbib compatible bst file
% argument is your BibTeX string definitions and bibliography database(s)
\bibliography{<%= filenames.bib %>}
<% break; case "scientific-thesis": -%>

\printbibliography
<% break; default: -%>

\bibliographystyle{plain}
\bibliography{<%= filenames.bib %>}
<% break; } -%>

% Enfore empty line after bibliography
\ \\
%
Alle Links wurden zuletzt am 29.03.2021 geprüft.

<% switch (documentclass) { case "lncs": -%>
%%% ===============================================================================
%\appendix
%\addcontentsline{toc}{chapter}{APPENDICES}

%\listoffigures
%\listoftables
%%% ===============================================================================

%%% ===============================================================================
%<%= heading1 %>{My first appendix}\label{sec:appendix1}
%%% ===============================================================================
<% break; case "scientific-thesis": -%>
%%% ===============================================================================
\appendix
\addcontentsline{toc}{chapter}{Anhang}

%\listoffigures
%\listoftables
%%% ===============================================================================

%%% ===============================================================================
%<%= heading1 %>{My first appendix}\label{sec:appendix1}
%%% ===============================================================================

\pagestyle{empty}
\renewcommand*{\chapterpagestyle}{empty}
\Versicherung
<% break; } -%>
\end{document}
