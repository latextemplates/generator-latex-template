<%= heading2 %>{Abkürzungen}

Mit \verb+\gls{...}+ können Abkürzungen eingebaut werden, beim ersten Aufrufen wird die lange Form eingesetzt.
Beim wiederholten Verwenden von \verb+\gls{...}+ wird automatisch die Kurzform angezeigt.
Außerdem wird die Abkürzung automatisch in die Abkürzungsliste eingefügt.
Mit \verb+\glspl{...}+ wird die Pluralform verwendet.
Möchte man, dass bei der ersten Verwendung direkt die Kurzform erscheint, so kann man mit \verb+\glsunset{...}+ eine Abkürzung als bereits verwendet markieren.
Das Gegenteil erreicht man mit \verb+\glsreset{...}+.

Definiert werden Abkürzungen in der Datei \textit{abbreviationstex} mithilfe von \verb+\newacronym{...}{...}{...}+.

Mehr Infos unter: \url{https://ctan.org/pkg/bib2gls}.

<%- bexample %>
Beim ersten Durchlauf betrug die \gls{fr} 5.
Beim zweiten Durchlauf war die \gls{fr} 3.
Die Pluralform sieht man hier: \glspl{er}.
Um zu demonstrieren, wie das Abkürzungsverzeichnis bei längeren Beschreibungstexten aussieht, muss hier noch \glspl{rdbms} erwähnt werden.

\gls{dante} is a local \TeX\ user group.
The German-speaking local \TeX\ user group is \gls{dante}.
A \gls{gp} is a medical doctor.
I went to my surgery to see the \gls{gp}.
<%- eexample %>
