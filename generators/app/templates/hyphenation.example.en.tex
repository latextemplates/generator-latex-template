<%= heading2 %>{Hyphenation}

\LaTeX{} automatically hyphenates words.
When using microtype, there should be less hypnetations than in other settings.
It might be necessary to tweak the hyphenations nevertheless.
Here are some hints:

<%- bexample %>
In case you write <% if (enquotes == 'csquotes') { %>\enquote{<% } else if (enquotes == 'textcmds') { %>\qq{<% } else { %>``<% } %>application-specific<% if ((enquotes == 'csquotes') || (enquotes == 'textcmds')) { %>}<% } else { %>''<% } %>, then the word will only be hyphenated at the dash.
You can also write \verb1applica\allowbreak{}tion-specific1 (result: applica\allowbreak{}tion-specific), but this is much more effort.

<% if (enquotes == 'csquotes') { -%>
In the net, you will read about that one can use a normal quote sign followed by an equal sign (\verb1"=1) to specify that hyphenation is allowed at other places in the word.
Since we use the quote sign for automatic quoation, this is not possible anymore.
Please use the command \verb1\diviswithhyphenation1 for the same purpose.
Example: \verb1application\diviswithhyphenation specific1 (result: application\diviswithhyphenation specific).
<% } else { -%>
You can now write words containing hyphens which are hyphenated (\verb1application"=specific1 gets application"=specific) at other places in the word.
This is enabled by an additional configuration of the babel package.
<% } -%>
<%- eexample %>
