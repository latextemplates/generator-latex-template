<%= heading2 %>{Hyphenation}

\LaTeX{} automatically hyphenates words.
When using \href{https://ctan.org/pkg/microtype}{microtype}, there should be fewer hyphenations than in other settings.
It might be necessary to tweak the hyphenations nevertheless.
Here are some hints:

<%- bexample %>
In case you write <%- bquote %>application-specific<%- equote %>, then the word will only be hyphenated at the dash.
You can also write \verb1applica\allowbreak{}tion-specific1 (result: applica\allowbreak{}tion-specific), but this is much more effort.

<% if (tweakouterquote == "outerquote")  { -%>
In the net, you will read about that one can use a normal quote sign followed by an equal sign (\verb1"=1) to specify that hyphenation is allowed at other places in the word.
Since we use the quote sign for automatic quotation, this is not possible anymore.
Please use the command \verb1\hydash1 for the same purpose.
Example: \verb1application\hydash specific1 (result: application\hydash specific).
<% } else { -%>
You can now write words containing hyphens which are hyphenated at other places in the word.
For instance, \verb1application"=specific1 gets application"=specific.
This is enabled by an additional configuration of the babel package.
<% } -%>
<%- eexample %>
