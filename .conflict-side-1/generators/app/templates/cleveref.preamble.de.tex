% cleveref für cref statt autoref, da cleveref auch bei Definitionen funktioniert
\usepackage[capitalise,nameinlink<% if (documentclass != 'lncs') { %>,noabbrev<% } %>]{cleveref}

<% if (documentclass == 'lncs') { -%>
\crefname{table}{Tabelle}{Tab.}
<% } else { -%>
\crefname{table}{Tabelle}{Tabellen}
<% } -%>
\Crefname{table}{Tabelle}{Tabellen}
\crefname{figure}{Abbildung}{Abbildungen}
\Crefname{figure}{Abbildung}{Abbildungen}
\crefname{equation}{Gleichung}{Gleichungen}
\Crefname{equation}{Gleichung}{Gleichungen}
\crefname{theorem}{Theorem}{Theoreme}
\Crefname{theorem}{Theorem}{Theoreme}
\crefname{listing}{Listing}{Listings}
\Crefname{listing}{Listing}{Listings}
\crefname{section}{Abschnitt}{Abschnitte}
\Crefname{section}{Abschnitt}{Abschnitte}
\crefname{paragraph}{Abschnitt}{Abschnitte}
\Crefname{paragraph}{Abschnitt}{Abschnitte}
\crefname{subparagraph}{Abschnitt}{Abschnitte}
\Crefname{subparagraph}{Abschnitt}{Abschnitte}
<% if (listings == 'minted') { -%>

% Intermediate solution for hyperlinked refs. See https://tex.stackexchange.com/q/132420/9075 for more information.
\newcommand{\llabel}[1]{\label[line]{#1}\hypertarget{#1}{}}
\newcommand{\lref}[1]{\hyperlink{#1}{\FancyVerbLineautorefname~\ref*{#1}}}
<% } -%>
