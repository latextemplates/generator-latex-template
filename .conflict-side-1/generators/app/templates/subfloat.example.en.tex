<%= heading2 %>{Sub Figures}

An example of two sub figures is shown in \cref{fig:two_sub_figures}.

<%- bexample %>
\begin{figure<% if (documentclass == 'ieee') { %>*<% } %>}[!b]
    \centering
    \subfloat[Case I]{\includegraphics[width=.4\linewidth]{example-image-a}%
    \label{fig:first_case}}
  \hfil
    \subfloat[Case II]{\includegraphics[width=.4\linewidth]{example-image-b}%
    \label{fig:second_case}}
  \caption{Example figure with two sub figures.}
  \label{fig:two_sub_figures}
\end{figure<% if (documentclass == 'ieee') { %>*<% } %>}
<%- eexample %>
<% if (documentclass == 'ieee') { -%>

Note that often IEEE papers with subfigures do not employ subfigure
captions (using the optional argument to \verb+\subfloat[]+), but instead will
reference/describe all of them (a), (b), etc., within the main caption.
Be aware that for subfig.sty to generate the (a), (b), etc., subfigure
labels, the optional argument to \verb+\subfloat+ must be present. If a
subcaption is not desired, just leave its contents blank,
e.g., \verb+\subfloat[]+.
An example is shown in \cref{fig:two_sub_figures_ieee}.

<%- bexample %>
\begin{figure*}[!b]
    \centering
    \subfloat[]{\includegraphics[width=.4\linewidth]{example-image-a}%
    \label{fig:first_case_ieee}}
  \hfil
    \subfloat[]{\includegraphics[width=.4\linewidth]{example-image-b}%
    \label{fig:second_case_ieee}}
  \caption{Example figure with two sub figures. IEEE style. (a) The first case. (b) The second case.}
  \label{fig:two_sub_figures_ieee}
\end{figure*}
<%- eexample %>
<% } -%>
