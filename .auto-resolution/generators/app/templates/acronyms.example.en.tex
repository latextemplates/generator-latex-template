<%= heading2 %>{Abbreviations}

With \verb+\gls{...}+ you can enter abbreviations, the first time you call it, the long form is used.
When reusing \verb+\gls{..}+ the short form is automatically displayed.
The abbreviation is also automatically inserted in the abbreviation list.
With \verb+\glspl{...}+ the plural form is used.
If you want the short form to appear directly at the first use, you can use \verb+\glsunset{..}+ to mark an abbreviation as already used.
The opposite is achieved with \verb+\glsreset{..}+.

Abbreviations are defined in \verb+\content\ausarbeitung.tex+ by means of \verb+\newacronym{...}{...}{...}+.

More information at: \url{https://ctan.org/pkg/bib2gls}.

<%- bexample %>
At the first pass the \gls{fr} was 5.
At the second pass was \gls{fr} 3.
The plural form can be seen here: \glspl{er}.
To demonstrate what the list of abbreviations looks like for longer description texts, \glspl{rdbms} must be mentioned here.

\gls{dante} is a local \TeX\ user group.
The German-speaking local \TeX\ user group is \gls{dante}.
A \gls{gp} is a medical doctor.
I went to my surgery to see the \gls{gp}.
<%- eexample %>
