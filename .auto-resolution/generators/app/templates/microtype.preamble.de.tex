% Optischer Randausgleich und Grauwertkorrektur. Siehe See http://www.ctan.org/tex-archive/macros/latex/contrib/microtype/

<% if (documentclass == 'acmart') { -%>
% microtype wird bereits von der ACM-Klasse geladen.
% Wir müssen "nur" ein paar Optionen anpassen.
<% } else { -%>
\usepackage[
  babel=true,
  expansion=alltext,
  protrusion=alltext-nott,
  final
]{microtype}
<% } -%>

% \texttt{test -- test} - diese Einstellung behält "--" bei (und konveriert sie nicht zu einem Bindestrich)
\DisableLigatures{encoding = T1, family = tt* }

% tracking=true muss als Parameter des microtype-packages mitgegeben werden
% Deaktiviert, da dies bei Algorithmen seltsam aussieht

%\DeclareMicrotypeSet*[tracking]{my}{ font = */*/*/sc/* }%

% Hier wird festgelegt, dass alle Passagen in Kapitälchen automatisch leicht gesperrt werden.
% Quelle: http://homepage.ruhr-uni-bochum.de/Georg.Verweyen/pakete.html
% Deaktiviert, da sonst "BPEL", "BPMN" usw. wirklich komisch aussehen.
% Macht wohl nur bei geisteswissenschaftlichen Arbeiten Sinn.
%\SetTracking{ encoding = *, shape = sc }{ 45 }
